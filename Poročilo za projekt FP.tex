\documentclass[a4paper, 11pt]{article}
\usepackage[slovene]{babel}
\usepackage[utf8]{inputenc}
\usepackage[T1]{fontenc}
\usepackage{amsfonts,amsmath,amssymb}
\usepackage{graphicx}
\usepackage{hyperref}
\usepackage{fancyvrb}
\setlength\parindent{0pt}

\newcommand{\R}{\mathbb R}
\newcommand{\N}{\mathbb N}
\newcommand{\Z}{\mathbb Z}
\newcommand{\C}{\mathbb C}
\newcommand{\Q}{\mathbb Q}

\begin{document}

\thispagestyle{empty}
\begin{center}
\begin{minipage}{0.75\linewidth}
    \centering
    {\Large Univerza v Ljubljani \\ Fakulteta za matematiko in fiziko}
    \\
    \vspace{7cm}

    {\uppercase{\LARGE \textbf{Bottleneck assignment in the plane}}} \\ Projekt pri predmetu Finančni praktikum \\
    \vspace{3cm}

    Avtorici:\\
    {\Large Lea Holc, Eva Rudolf\par}
    \vspace{7cm}

    {\Large Ljubljana, 2023}
\end{minipage}
\end{center}


\newpage
\tableofcontents
\newpage

\section{Navodilo}
Naj bosta dani dve množici točk v ravnini $P$ in $Q$. Elementi množice $P$ predstavljajo proizvajalce 
(\textsl{providers}), elementi množice $Q$ pa porabnike (\textsl{clients}). Tako proizvajalci kot porabniki 
upravljajo z isto dobrino. Vsak proizvajalec $p \in P$ lahko proizvede $s_p > 0$ dobrine, za vsakega porabnika 
$q \in Q$ pa definiramo njegovo povpraševanje po dobrini kot $d_q > 0$. Predpostavimo, da je ponudba vseh 
prozvajalcev enaka povpraševanju vseh porabnikov. Cilj projekta je ugotoviti, ali za podano vrednost $\lambda$ 
proizvajalci lahko zadostijo potrebam porabnikov in sicer tako, da vsaka dobrina prepotuje pot največ $\lambda$. 

\section{Linearen program}
Ukvarjamo se s problemom maksimalnega pretoka, ki ga želimo predstaviti kot linearni program (LP).
Cilj je bil napisati linearni program, ki minimizira največjo razdaljo, ki jo mora prepotovati dobrina 
od proizvajalca do porabnika, pri čemer moramo zadostiti vsem potrebam, ki jih imajo porabniki ne da 
bi presegli kapacitete, ki jih lahko proizvede posamezen proizvajalec. 
\subsection{Oznake}
Najprej uvedemo oznake. Naj bo $p_i$ za $i=1,\dots,m$ posamezen proizvajalec s koordinatami $(x_i,y_i)$ in $s_i$ za $i=1,2,\dots,m$ njegova proizvodnja dobrine. 
Naprej naj bo $q_j$ za $j=1,2,\dots,n$ posamezen porabnik s koordinatami $(w_j,z_j)$ in $d_j$ za $j=1,2,\dots,n$ njegovo povpraševanje 
po dobrini. Spomnimo se, da tako proizvajalci kot porabniki upravljajo z isto dobrino in predpostavimo, da velja
$$
\sum_{i=1}^m s_i = \sum_{j=1}^n d_j.
$$ 
S $c_{ij}$ označimo količino, ki jo proizvajalec $i$ dobavi porabniku $j$. Ker vsi proizvajlci ne 
dobavljajo vsem porabnikom, uvedemo še indikatorsko spremenljivko $b_{ij}$.
$$
b_{ij} = 
\begin{cases}
    0;\qquad \textrm{i-ti proizvajalec ne oskrbuje j-tega ponudnika} \\
    1;\qquad \textrm{i-ti proizvajalec oskrbuje j-tega ponudnika} \\
\end{cases} 
$$
Z $\lambda$ označimo maksimalno razdaljo, ki jo lahko prepotuje posamezna dobrina. To razdaljo 
želimo minimizirati.
\subsection{Pogoji}
Zapišimo še pogoje, s katerimi zadostimo potrebam posameznega porabnika in zmožnostim posameznega 
proizvajalca:
$$
\sum_{j=1}^n c_{ij} = s_i \text{za} \forall i = 1,\dots,m 
$$
$$
\sum_{i=1}^m c_{ij} = d_j \text{za} \forall j = 1,\dots,n 
$$
Za indikatorsko spremenljivko $b_{ij}$ lahko zapišemo $$0 \leq b_{ij} \leq 1 \text{in} b \in \Z.$$
Veljati mora tudi $b_{ij}=0 \Rightarrow c_{ij}=0$, kar lahko zapišemo kot 
$$0 \leq c_{ij} \leq c_{ij}\cdot b_{ij}.$$
Omejimo še $\lambda$:
$$b_{ij}\cdot \sqrt{(x_i - w_j)^2+(y_i - z_j)^2} \leq \lambda.$$

Naš cilj bo poiskati:
$$
\min \max_{i,j} \sqrt{(x_j-x_i)^2 + (y_j-y_i)^2} \cdot c_{ij}.
$$

\section{Delo na projektu}
Glede na to, da se ukvarjamo z linearnim programiranjem, sva uporabili programski jezik Sage.
\end{document}


\end{document}