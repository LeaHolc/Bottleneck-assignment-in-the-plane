\documentclass[a4paper, pt14]{article}

\usepackage[utf8]{inputenc}
\usepackage[T1]{fontenc}
\usepackage[slovene]{babel}
\usepackage{lmodern}
%\usepackage{hyperref}
\usepackage{amsmath}
\usepackage{amssymb}

\begin{document}

\title{%
  Bottleneck assignment in the plane \\
  \large projekt pri pri predmetu Finančni praktikum}
\author{Lea Holc in Eva Rudolf}
\date{16. \ 12. \ 2022}

\maketitle

\section{Navodilo naloge}
Naj bosta dani dve množici točk v ravnini $P$ in $Q$. Elementi množice $P$ predstavljajo proizvajalce (\textsl{providers}), elementi množice $Q$ pa porabnike (\textsl{clients}). Tako proizvajalci kot porabniki upravljajo z isto dobrino. Vsak proizvajalec $p \in P$ lahko proizvede $s_p > 0$ dobrine, za vsakega porabnika $q \in Q$ pa definiramo njegovo povpraševanje po dobrini kot $d_q > 0$. Predpostavimo, da je ponudba vseh prozvajalcev enaka povpraševanju vseh porabnikov. Cilj projekta je ugotoviti, ali za podano vrednost $\lambda$ proizvajalci lahko zadostijo potrebam porabnikov in sicer tako, da vsaka dobrina prepotuje pot največ $\lambda$. Ukvarjamo se s problemom maksimalnega pretoka, ki ga želimo predstaviti kot linearni program (LP).

\end{document}